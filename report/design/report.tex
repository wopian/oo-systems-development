\chapter{Report}

% Submit a brief report to accompany the above, including an
% explanation of the process used to produce the above models
% stating any constraints and assumptions used along with any
% difficulties encountered and the course of action taken to
% overcome them.  Also include an appendix recording team meetings
% and discussions.


\section{Process used}
We analysed the study case first and tried to obtain the use cases first so that we could know what exactly the application would have to do. After that, we started to think about how we could implement that functionality and that's how we started to do the class diagrams, since the implementation thinking process helped us identify necessary classes, operations and attributes.

\section{Constraints and Assumptions}
In order to produce the model, we did some assumptions:
\begin{itemize}
  \item The payments would always be accepted
  \item An event represents a performance that runs in the theatre for a certain period of time (for example, a musical that is being played at the theatre for two months)
  \item A show is an individual performance of an event (continuing in the musical example, a show would be a single performance of the musical occurring in an evening)
\end{itemize}

\section{Difficulties Encountered}
The major encountered difficulties we had was with identifying how the system would work and how all the classes would be linked to each other, as well as what each one would do (responsibility assignment). In order to overcome it, we decided to modulate the operations as much as possible and get every interested party to do a little bit of work towards its completion.

\section{Contribution and discussions}
The log of contributions and discussions from the team can be found in the appendix at the end of the report.
